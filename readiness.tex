\section{Rubin Operations Team Readiness} \label{sec:ops}

\subsection{Operations Readiness Requirement}

\begin{itemize}

	\item The Operations Team shall have a detailed operations plan approved by NSF and DOE.
	\item The Operations Team shall have a staffing plan with all roles in the operations plan filled with identified personnel. 
	\item The Operations Team shall demonstrate they can operate the delivered Rubin System
to efficiently capture, store, and process science quality images.
	
\end{itemize}

\subsection{Objectives}

The primary objective of this element of the ORR is that the Operations Team demonstrates that it 
is ready to smoothly continue running the full Rubin System as it exists at the end of the 
commissioning period. A successful initial phase of operations may include beginning the full 
Legacy Survey of Space and Time at the approved nightly schedule and cadence. 
It may also include other activities as necessary depending on the final outcome of commissioning. 
These could include special observing modes to enable Early Science and further 
development of detailed procedures for operations not done in commissioning but which 
do not prevent completion criteria from being satisfied.  

\subsection{Criteria for Readiness}

\begin{itemize}

	\item Demonstrate planning and staff for safe operations are in place. 
	\item The team should demonstrate that all needed roles are filled, or will be, with 
trained staff at the time of hand over to full operations.
	\item All Human Resources processes for on-boarding operations staff should be complete 
or ready by the date of handover as appropriate. Expatriate staff for Chile based deployments 
should have all necessary documents and requirements for work in Chile in place. Chilean staff 
should have any needed changes to their contracts made before operations begin. 
	\item An operations budget profile fully covering the needs of the observatory should be 
agreed to with the agencies in advance of full operations beginning.
	\item All supplies and non-labor capital items should be in place. 
	\item Contracts needed in year 1 for operations services or supplies should be in place. 
	\item Any in-kind contributions necessary for operations should be demonstrated to be in 
place and functioning at the level needed for year 1. Any systems handed over to operations 
from construction in advance of this review should be demonstrated to be functioning at the 
required level of performance.
	\item Demonstrate all needed advisory committees/structures are ready and in place.
	\item Demonstrate that all construction related documentation is captured 
in an operations documentation management system.
	\item Demonstrate ability to execute Alert Processing in the US DF including connectiing to community brokers.
	\item Demonstrate ability to execute Data Release Processing including delivery of data to
non US DF Data Facilities and ingest of data products from same for Data Access at USDF and Chile
DAC.  
	\item Demonstrate that a significant fraction of the community has been granted 
user accounts in the US DF, that the Rubin Science Platform supports their access and 
authorization and that they have been given suitable training or information to do science 
with the Rubin data products as they are delivered.

\end{itemize}

\subsection{Artifacts for ORR}

 As prelude: the Construction team will be responsible for creating sets/lists of topics/documents that fully describe the characteristics and performance of the Rubin systems, how to maintain them, how to operate them, and anything else critical for the Operations Team (initial survey of documents suggested date November 2020. The Operations Team will review these lists and identify anything that needs to be added (or removed) from those lists. A collaborative negotiation will be carried out with the Construction Team.

Final managing organization and agency approved Detailed Observatory Operations Plan, including:

\begin{itemize}
	\item Work Breakdown Structure;
	\item Activity based plans for each department;
	\item Milestones for each department though several years of operations;
	\item Performance metrics;
	\item Performance requirements;
	\item Maintenance Management plans;
	\item Fully populated staffing plan;
	\item Budget profile; and
	
\end{itemize}
