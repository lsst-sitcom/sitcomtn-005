
\section {Introduction}

One of the primary high-level strategic inputs to developing the System AI\&T and Commissioning Plan (\citeds {LSE-79}) are the construction completeness requirements for the Operations Readiness Review (ORR). At the conclusion of the Commissioning Phase of the LSST construction project an ORR will be undertaken by an external panel, jointly appointed by the DOE and NSF, in consultation with the LSST Project Team. The ORR also signifies the end of the NSF MREFC funded construction project and DOE Commissioning.

We collect together the elements that constitute criteria for Operations Readiness. Each topic has, or will have, defined requirements -- in many cases along with goals and stretch goals –  that will each have the relevant documentation for performance against the requirement. For those requirements that specify performance after some period of operations, the basis of estimate of projected performance will be provided. Unless otherwise specified, functional requirements will be verified by direct test, and performance requirements will be verified by direct test, analysis, or a some combination thereof. For each requirement, there will either be a clean pass, or there will be a waiver process that documents why it is acceptable to proceed to operations (or the reason we must postpone the transition to operations).

Some of the topics are already covered by existing verification plans. Some functional requirements (and any accompanying goals and stretch goals) are still in review -- in those cases, the requirements and associated verifications are being developed together to ensure clarity and verifiability. Some topics, such as the mini-survey, have requirements that are a combination of performance and functionality that do not necessarily flow directly from the high-level requirements; in those cases, we identify the minimum requirements that must be met to proceed to operations, along with a range of goals and stretch goals and the accompanying rationale.