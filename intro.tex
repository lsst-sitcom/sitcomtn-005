
\section {Introduction}

One of the primary high-level strategic inputs to developing the System AI\&T and Commissioning Plan (\citeds {LSE-79}) are the construction completeness requirements for the Operations Readiness Review (ORR). At the conclusion the Rubin Observatory Construction Project's commissioning phase an ORR will be undertaken by an external panel, jointly appointed by the DOE and NSF, in consultation with the Project Team. The successful completion of the ORR will signify the end of the NSF MREFC funded construction project and DOE Commissioning.

The ORR will consist of two parts: 1) The evaluation of the Rubin Construction Project completeness and 2) the readiness of Rubin Observatory Operations team's readiness to receive the construction deliverables and begin planned operations for conducting the Legacy Survey of Space and Time -- the 10-year science survey for which the Rubin Observatory was designed and constructed to perform.

In this document, we collect together the elements that constitute criteria for construction completeness and operations readiness. Each topic has it own, or will reference, well defined requirements -- in some cases these include goals and stretch goals -- each will have the relevant supporting documentation for performance against the requirement. For those requirements that specify performance after some period of operations, the basis of estimate of projected performance will be provided. Unless otherwise specified, functional requirements will be verified by direct test, and performance requirements will be verified by direct test, analysis, or a some combination thereof.  For each requirement, there will either be a clean pass, or there will be a waiver process that documents why it is acceptable to proceed to operations (or the reason we must postpone the transition to operations).

Some topics summarized in this document are already covered by existing verification plans.  Some functional requirements (and any accompanying goals and stretch goals) are still in review (at the time of this document version) -- in those cases, the requirements and associated verifications are being developed together to ensure clarity and crisp requirements for verifiability. Some topics, such as the Science Validation surveys, have requirements that are a combination of performance and functionality that do not easily flow directly from the high-level system requirements; in those cases, we identify the minimum requirements and performance that must be met to proceed to operations, along with a range of goals and stretch goals and the accompanying rationale.