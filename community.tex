\section{Verification of Education and Public Outreach}  \label{sec:community}

\subsection{Operations Readiness requirement}

In order for the Rubin Observatory program to declare that the construction is complete and is ready to enter its Operations Phase, the Project shall demonstrate that EPO program elements have been verified against requirements, the interfaces aimed at the general public are functional and accessible, and content is sufficiently populated to represent Rubin Observatory and its services.

\subsection{Objectives}

The objectives of this Operational Requirement are to ensure that the public-facing interfaces are functional and accessible by members of the general public. These include the Education Hub, news pages, multimedia gallery, and Citizen Science infrastructure. Additionally, the Communications Strategy should be documented and the EPO Data Center should be functional.

\subsection{Criteria for Completenes}

The following breaks down the overall EPO Program into distinct elements with associated completeness descriptions:

\subsubsection{Citizen Science}

At completion, researchers who want to lead citizen science projects with Rubin Observatory data can create a sample set using the tools in the Rubin Science Platform (RSP) with whatever data is available at the time.

Rubin Observatory users will be able to create citizen science projects with any LSST data.  At completion, we will have demonstrated that:
\begin{itemize}
	\item Users can use the tools in the Rubin Science Platform (RSP) with whatever data is available at the time then move data to the Zooniverse Project Builder Tool, with applicable data rights observed.
	\item This procedure is successful having tested two citizen science projects following this workflow.
\end{itemize}

\subsubsection{Website}

The public-facing website will be ready and live.  The EPO team will have demonstrated that at minimum the following functions are operable:

\begin{itemize}
	\item The Rubin Observatory EPO website featuring:
	\begin{itemize}
		\item A News page;
		\item the Skyviewer;
		\item A multimedia Gallery;
		\item Staff profiles,
		\item Ready to highlight features from the Alert Stream; and
		\item Relevant material from the existing lsst.org pages will have been migrated to the new site.
	\end{itemize}
	\item The Skyviewer as an interactive page allowing the display of color images over  large patches of sky and allows users to pan and zoom, and that the Skyviewer features at least one tour of astronomical objects relevant to Rubin science goals;
	\item The Multimedia Gallery featuring free assets that follow AVM metadata standards:
	\begin{itemize}
		\item A set of videos for Planetarium use;
		\item Image highlights and a virtual tour of Rubin Observatory; and
		\item A short videos describing Rubin science and facilities.
	\end{itemize}
\end{itemize}

\subsubsection{Formal Eduction}

The Formal Education Program offers a suite of online investigations that are web applications where users interact with astronomical data via widgets. The investigations and educator support materials will be accessible through the “Education Hub.”   At completion, the EPO team will have demonstrated that:

\begin{itemize}
	\item The investigations and educator support materials are accessible through the Education website;
	\item Documentation describing the Professional Development plan for educators is completed.
	\item Infrastructure for providing education materials in Spanish language is complete.
\end{itemize}

\subsubsection{EPO Data Center}

At completion, the EPO team will have demonstrated that the EPO Data Center is cloud-based and is serving data to the EPO website and products.

\subsection{Pre--Operation Interactions}

The final delivered infrastructure and documentation will be negotiated between the Rubin Construction Project and NOIRLab.

\subsection{Artifacts for Completion}

The EPO Team will provide evidence of verifying requirements in the Jira system and provide general documentation about each part of the program described above.


