\documentclass[SE,lsstdraft,authoryear,toc]{lsstdoc}
% lsstdoc documentation: https://lsst-texmf.lsst.io/lsstdoc.html
\input{meta}

% Package imports go here.

% Local commands go here.

%If you want glossaries
%\input{aglossary.tex}
%\makeglossaries

\title{Construction Completeness and Operations Readiness Criteria}

% Optional subtitle
% \setDocSubtitle{A subtitle}

\author{%
Chuck Claver, 
Keith Bechtol, 
Eric Bellm,
Robert Blum, 
Jim Bosch, 
Andy Clements,
Andrew Connolly, 
Leanne Guy,  
\v{Z}eljko Ivezi\'{c}, 
Robert Lupton,
Steve Ritz,
William O'Mullane, and
Sandrine Thomas
}

\setDocRef{SITCOMTN-005}
\setDocUpstreamLocation{\url{https://github.com/lsst-sitcom/sitcom-005}}

\date{\vcsDate}

% Optional: name of the document's curator
% \setDocCurator{The Curator of this Document}

\setDocAbstract{%
This document collects together the elements that constitute the criteria for completeness of the Rubin Observatory MREFC Construction Project,  DOE Rubin Observatory Commissioning, and the readiness for Rubin Observatory operations to conduct the 10--year Legacy Survey of Space and Time (LSST).
\bigskip

This is a living document and will be modified and refined as required throughout the remainder of the combined NSF -- DOE Rubin Construction project.
\bigskip

In addition to this document and references therein, the completion of the Rubin Observatory Project will be evaluated based on the LSST Project Execution Plan (\citeds {LPM-17}) and the Commissioning Execution Plan (\citeds{LSE-390}).  The completeness evaluation will be done through a joint NSF and DOE Operations Readiness Review having two parts: 1) A review of the construction Project's meeting its requirements as outline in this document and 2) a review of the Rubin Operations team's readiness to begin the 10-year Legacy Survey of Space and Time (LSST).
}

% Change history defined here.
% Order: oldest first.
% Fields: VERSION, DATE, DESCRIPTION, OWNER NAME.
% See LPM-51 for version number policy.
\setDocChangeRecord{%
  \addtohist{1}{2020-08-06}{First draft}{Leanne Guy}
  \addtohist{1}{2020-08-20}{Initial version for internal review}{Chuck Claver}
  }


\begin{document}

% Create the title page.
\maketitle
% Frequently for a technote we do not want a title page  uncomment this to remove the title page and changelog.
% use \mkshorttitle to remove the extra pages

% ADD CONTENT HERE
% You can also use the \input command to include several content files.


\section {Introduction}

There are two main criteria to evaluate the readiness of the as-built Vera C. Rubin Observatory:

\begin{enumerate}

        \item completion of the Rubin Observatory Construction Project and

        \item readiness of Rubin Observatory Operations to recieve the construction deliverables and begin the Legacy Survey of Space and Time (LSST) -- the 10-year science survey for which the Rubin Observatory was designed and constructed to perform.

\end{enumerate}

These two main considerations have been expanded into 10 points of Construction Completeness and Operations Readiness as defined in the Rubin Observatory {\it System AI\&T and Commissioning Plan} (\citeds{LSE-79}):

%One of the primary high-level strategic inputs to developing the {\it System AI\&T and Commissioning Plan} (\citeds{LSE-79}) is the set of construction completeness requirements.
% for the Construction Closeout Reviews (CCRs).
%In (\citeds{LSE-79}), the Project has identified 10 general requirements for "construction completeness", including one requirement for "operations readiness", summarized as:

\begin{enumerate}
        \item Verification of LSST System Requirements (\citeds {LSE-29}) and survey performance as described in SRD (\citeds {LPM-17})
        \item Verification of the Observatory System Specifications (\citeds {LSE-30})
        \item Verification of Data Processing, Products and User Services
        \item Demonstrating Science Data Quality Assessment (SDQA)
        \item Conduct a Science Validation Survey
        \item Demonstrate the system state is recorded and archived for each observation
        \item Verify Education and Public Outreach has met its requirements and construction scope
        \item Operational procedures and documented and accessible
        \item Provided a record of the as-built system, including modification since the as-build and non-compliance
        \item Demonstrate Rubin Operations Team readiness.
\end{enumerate}

At the concluding stages of the Rubin Observatory Construction Project's commissioning phase, a series of four Construction Closeout Reviews (CCRs) will be undertaken by an external panel jointly appointed by the DOE and NSF in consultation with the Project Team.
The successful completion of the CCRs will signify the end of the NSF MREFC-funded construction project and DOE Commissioning.
The CCRs are consistent with the NSF guidance given in the {\it Major facilities Guide} (\citeds{NSF-19-68}) Sections 2.4.2.1 -- {\it Project Close--out Process}, 3.4.2.15 -- {\it Commissioning}, 4.4 -- {\it System Integration, Testing and Acceptance}, and 4.5 -- {\it Documentation Requirements}.
The expected timeline and focus areas for the four CCRs are summarized in Section~\ref{ccr}.
%Figure~\ref{CCRs_overview}.

%There will be four reviews in the series covering

%\begin{enumerate}
%        \item evaluation of the Rubin Construction Project completeness criteria and
%        \item the Rubin Observatory Operations team's readiness to receive the construction deliverables and begin planned operations for conducting the Legacy Survey of Space and Time -- the 10-year science survey for which the Rubin Observatory was designed and constructed to perform.
%\end{enumerate}

In this document, we collect and detail the elements that constitute the criteria for construction completeness and operations readiness.
Each topic has its own and/or references well-defined requirements -- in some cases, these include goals and stretch goals -- each will have the relevant supporting documentation for performance against the requirement.
For those requirements that specify performance after some period of operations, the basis of the estimated projected performance will be provided.
Unless otherwise specified, functional requirements will be verified by direct test, and performance requirements will be verified by direct test, analysis, or some combination thereof.
For each requirement, there will either be a clean pass or a waiver process that documents why it is acceptable to proceed to operations (or the reason we must postpone the transition to operations).

Some topics summarized in this document are already covered by existing verification plans.
Some functional requirements (and any accompanying goals and stretch goals) are still in review (at the time of this document version) -- in those cases, the requirements and associated verifications are being developed together to ensure clarity and crisp requirements for verifiability.
Some topics, such as the Science Validation surveys, have criteria that are a combination of performance and functionality that do not easily flow directly from the high-level system requirements; in those cases, we identify the minimum criteria and performance that must be met to proceed to operations, along with a range of goals and stretch goals and the accompanying rationale.

For each of the general construction completeness requirements, we provide:

\begin{itemize}
	\item the statement of the requirements;
	\item an expansion of objective and intent;
	\item specific criteria for completeness;
	\item indication of any pre--Operation interactions; and
	\item the expected delivered artifacts.
\end{itemize}

\section{LSST System Requirements \& SRD Verification/Validation}  \label{sec:srd}

\section{Observatory System Specifications (LSE-30) Verifcation}  \label{sec:oss}

\subsection{Operations Readiness Requirement}
The project team shall demonstrate that the integrated LSST systems (Camera, Telescope \& Site and Data Management subsystems) as well as the Education and Public Outreach (EPO) system have met the technical specifications enumerated in the LSST Observatory System Specifications (\citeds{LSE-30}).

%new_hd
The requirements in \citeds{LSE-30} have been marked according to the CCR where they can be earliest verified. 
The distribution between the CCRs is shown in Figure~\ref{CCRs_overview}.

\begin{figure}[htbp]
\begin{center}
%\includegraphics[width=1\textwidth]{./XXX.png}
\caption{Distribution of the the LSE30 requirement verification over the course of the CCRs}
\label{LSE30_CCRdistribution}
\end{center}
\end{figure}


\subsection{Objectives}
The main objective of this Operations Readiness Requirement is to verify the system specifications in the OSS (\citeds{LSE-30}) are proven and well documented. The OSS is essentially the highest-level document describing the basic LSST system technical architecture. It contains sections derived from the LSR on the following broad topics:

\begin{itemize}
\item System Composition and Constraints

\item Common System Functions and Performance, including:

	\begin{itemize}
		\item System Control
		%new_hd
		The System Control is implemented by combining a Service Abstraction Layer (SAL) and a number of Commandable SAL Components (CSC).
		A CSC represents each System and Subsystem in the observatory.
		Each CSC has a well-defined interface with the SAL. All other CSCs are required to comply with the definition of the interface.
		Therefore, the interface definitions are handled as requirements and verified as such. 
		Each interface requirement is verified through unit testing on the teststands at each new release and with the hardware during system usage.  
		Artifact?
		
		\item System Monitoring and Diagnostics
		%new_hd
		As part of the communication between the CSCs, messages with Commands, Events, and Telemetry are exchanged. 
		These are stored in real-time in the Engineering database and can be displayed through Chronograph, Rubin TV, and others.
		To verify these efforts, we demonstrate the capabilities during the observatory visit. 
				
		\item System Maintenance
		%new_hd
		Maintenance started as soon as the Observatory started to use components that needed maintenance, such as generators.
		We have implemented a Computerized Maintenance Management System (CMMS) and connected it to our work management system (Jira) 
		
		\item System Availability
		%new_hd
		The system availability depends on several technical aspects. Principally power and cooling. 
		We have a staged system with the national grid as a primary power source to ensure power. 
		As a backup, we have three power levels with decreasing capabilities: two generators and UPS batteries.
		Cooling consists of redundant Chillers and pumps that can make the best use of the cooling power stored in the system.
		At CCR1, the power and cooling installations are presented. 
		%Present ongoing efforts to improve the switch between power source distribution and syncing.
		
		\item System Time References
		%new_hd
		For the time reference, we have a local time server connected to the internet providing high precision time reference at any given moment.  		
		
	\end{itemize}

\item Detailed Specifications:

	\begin{itemize}
		\item Science and Bulk Data
		
		\item Optical System 
		%new_hd
		The optical system consists of the three mirror surfaces, the camera lenses, and the detectors. Each element has been tested indiv
		idually. At CCR1, we present an overview of the artifacts collected during the fabrication and coating processes.
		
		\item System Throughput
		
		This is addressed in the SRD section.
		
		\item Camera System
		The LSSTCam is still in verification during the time of the CCR1
		We will present the actual state of the testing, integration, and commissioning activities and a plan to finalize the commissioning.
		
		\item Photometric Calibration
		The calibration system is still being verified during the time of the CCR1.
		We will present the actual state of the testing, integration, and commissioning activities and a plan to finalize the commissioning.
		
		\item System Timing and Dynamics
		We present the status of the TMA testing and integration with the attached subsystems.
		
	\end{itemize}
	
\item Education and Public Outreach
	EPO has already entered operations. During CCR1, we briefly present their status.
	
\end{itemize}

\subsection{Criteria for Completeness}
Compliance with this objective will follow the process defined in the Verification and Validation Process document (\citeds{LSE-160}) and associated documentation.  
All technical specifications in the OSS (\citeds{LSE-30}) and LSR (\citeds{LSE-29}) are expected to be met at the end of construction.

\subsection{Pre--Operations Interaction}
None. Unless there are non-compliance issues with the ORR requirements and specifications.

\subsection{Artifacts for Completion}

\begin{itemize}
 
	\item Verification matrix containing entries for all OSS requirements and specifications.  The verification method: inspection, demonstration, analysis or test shall be identified for every OSS requirement.  Final compliance status will be included.
	\item Analysis reports where the verification method has been identified as "test" or "analysis".
	\item Non-compliance reports.

\end{itemize}
\section{Verification and Validation of Data Management}  \label{sec:dm}


\subsection{Verification Procedure for Data Management System Requirements (LSE-61)}


\subsection{Prompt Processing}


\subsection{Data Release Processing}


\subsection{Rubin Science Platform}

\section{Science Data Quality Assessment}  \label{sec:sdqa}


\subsection{Operations Readiness Requirement}

The project team shall demonstrate that the integrated LSST system can monitor and assess the quality of all data as it is being collected.

\subsection{Objectives} 

Science Data Quality Assessment is made up of a comprehensive system of tools to monitor and assess quality of all data as it is being collected including raw and processed data. The suite of tools have been designed to collect, analyze and record required information to assess the data quality and make that information available to a variety of end users; observatory specialist, observatory scientists, downstream processing, the science planning/scheduling process and science users of the data. 

The fast cadence of data collection requires highly automated data diagnostic and analysis methods (such as data mining techniques for finding patterns in large datasets, and various machine learning regression techniques). he Science Data Quality Assessment is mostly be automated, however it includes human-intensive components allowing further investigation and visualization of SDQA status.

Data quality assessment for Rubin must be carried out at a variety of cadences, which have different goals:

\begin{itemize}

	\item Near real-time assessment of whether the data is scientifically useful;
	\item Monitoring telemetry and imaging data to track the state of the integrated observatory, including the telescope, camera, networks and other supporting systems;
	\item Analysis of the prompt processing properties and performance to determine if the alerts stream meets its requirements; and
	\item Analysis of the data release processing properties and performance to determine if the static sky processing meets its requirements.
	
\end{itemize}

By the time we make a data release the accumulated data quality analysis must be made available as part of the release artefacts.

\subsubsection{Near Real--time Monitoring \& Assessment of the raw data quality} 

The quality assessment of the raw data combines the results from the state of the telescope, the camera (see below) and technical properties of the images.  We will analyze each image as it is taken to a measure its properties both on the at the Summit Facility using the LSSTCam Diagnostic cluster and from properties determined during the prompt processing for alert production.  Performance properties will be based on measurements and characteristics derived from the images themselves and from daily calibration data, these include:

\begin{itemize}

	\item Readnoise, bias stability, gain variations, bitwise integrity etc...  from the CCD data;
	\item Shape of the PSF, based on the three second moments, or equivalently effective FWHM, e1, e2;
	\item Sky background level over the FPA;
	\item Source position	sb relative to a reference catalog ({\it e.g. GAIA}) to monitor FPA stability and pointing acuracy; and
	\item Source brightness relative to a reference catalogue ({\it e.g. GAIA}) to monitor system throughput and sensitivity.
	
\end{itemize}

Together, these data enable us to determine if the data are within performance parameters to label the visit as "good".   Tooling will be provided by the construction project to allow users to monitor trends in these quantities (e.g. as a function of time and where the telescope is pointing;  as a function of position in the focal plane).  These will initially be provided by the LOVE interface (see below), although more detailed analysis may require additional tooling.  In some cases, data from the Rubin Auxiliary Telescope (RAT) may be used to help interpret trends discovered in the LSSTCam data.  Not discussed here is the quality analysis needed to determine that the RAT is taking sufficiently good data.

\subsubsection{Longer Term Assessment}

TBD

\subsubsection{Assessing the quality of the processed data}

The information of the processed data relies on the calibration data products and the pipeline properties. In other words, the data assessment at this stage shall include the correction of the systematic errors. 

\subsection{SDQA Tools for analysis}

Science Data Quality Assessment will rely on a suite of tools including as the electronic logging, the engineering facility database (EFD), and the Rubin Science Platform (RSP).  There is also a complementary set data visualization tools to facilitate the understanding of the correlation between the data quality and the observatory state. 

These tools include:

\begin{itemize}

	\item Rubin Science Platform (RSP) -- used for investigative ad--hoc analysis (\citeds{lse-319});  the RSP itself through it's web based porthole and Jupyter Lab interface provides significant visualization capabilities;
	\item Engineering Facility Database -- accessible through science platform and pre-defined dashboards;
	\item LOVE - LSST Observing Visualization Environment used to have standardized dashboards and visualization of the system state;
	\item SQuaSH - the Science Quality System Harness (\citeds{sqr-009})

\end{itemize}

\subsection{Criteria for Completeness Description}

The SDQA shall monitor and record the properties of the system error budget tree, including image quality and throughput, and define pass or fail status at each of the primary entries entries.   These include the following terms of the image quality: 

\begin{itemize}

	\item PSF FHWM;
	\item PSF shape ellipticity as described by second moments;
	\item System wavefront measurements for each visit; and
	\item Throughput measurements over the entire field of view.
		
\end{itemize}

Tooling for evaluating SDQA shall demonstrate the ability to display performance on a visit by visit basis as well as being able to show the history of performance metric over a user defined span of time.

\subsection{Pre-Operations Interactions}
The pre-operation interaction include training the observing specialists to understand errors 

\subsection{Artifacts for ORR}

\begin{itemize}

	\item Demonstrated functional tool kit as described above;
	\item Code validation tool kit to quantify software performance;
	\item Derived reporting from the Science Verification/Validation survey(s)
	
\end{itemize}






\section{Science Validation Survey}  \label{sec:svs}

\section{Recording and Archiving of System State Metadata}  \label{sec:metadata}

\subsection{Operations Readiness Requirement}

The Rubin Project Team shall demonstrate that relevant metadata are being collected and archived.

\subsection{Objectives:}

The objective with this requirement is to ensure that the technical state of the environment and hardware/software systems during the time of survey data collection is recorded with sufficient fidelity to be used in support of subsequent processing to produce the LSST science products. This is of particular importance for the determination and correction of systematics in the data as the survey progresses and statistics improve.  Additionally, the metadata record in required to assure efficient operation and maintenance of the observing facility.   The primary repository of this metadata is the Engineering Facility Database (EFD) - having two components: 1) a searchable SQL Cluster based capture of "house keeping" telemetry and 2) the Large File Annex for non-telemetry records (e.g. configuration files, images, other binary files outside the science pixel data etc...).

Technical Metadata at the time of each visit includes but not limited to:

\begin{itemize}

	\item Meteorological state on the Summit;
	\item Environmental conditions in the dome interior;
	\item Atmospheric seeing as measured by the tower mounted DIMM;
	\item Sky transparency map from the All-Sky Camera;
	\item Technical "house keeping" telemetry from each subsystem component as published to the EFD;
	\item Software version configuration status of all operating systems; and
	\item Configuration parameters of all active subsystems.
	
\end{itemize}

\subsection{Criteria for Completeness}

Satisfying this criteria includes at a minimum:

\begin{itemize}

	\item Demonstrate the technical data (see above) are being recorded by the EFD at >99\% (TBC) reliability level - e.g. no significant dropouts in the live database at the Summit Facility;
	\item Demonstrate the recorded data are being archived for long term access - copy at Base Facility in Chile and Copy at NCSA (possibly Interim Data Facility);
	\item Access to the technical data is achievable through standard monitoring dashboards;
	\item Access to the technical data is chewable through use customizable GUI interface(s); and
	\item Technical data are queryable through Rubin Science Platform tools - e.g. Jupyter Lab notebooks and WEB interface.
	
\end{itemize}

\subsection{Pre--Operations Interactions}

Transfer and archiving the EFD at the Interim Data Center would be required for external queries.

\subsection{Artifacts for ORR}

\begin{itemize}

	\item Report documenting minimum criteria as defined in the discussion section above
	\item SDK and example code for custom dashboards and dashboard templates available through software repository(s) - e.g. GitHub
	\item Example code for Rubin Science Platform queries through software repository - e.g. GitHub
	
\end{itemize}
\section{Verification of Education and Public Outreach}  \label{sec:community}

\section{Operational Procedures} \label{sec:docs}

\subsection{Operations Readiness Requirement}

The project team shall deliver a complete set of documented operational procedures and supporting technical documents needed to operate the LSST as a scientific facility to conduct a 10-year survey.

\subsection{Objectives:}

The objective of this Operational Requirement is to ensure that the procedures necessary for the operations and maintenance of the Rubin Observatory are documented and provided in a form that allows the operations team to conduct the 10-year planned survey. The documentation is to include but is not limited to:


\subsection{Criteria for Completeness}

The documentation is to include but is not limited to:

\begin{itemize}
	\item Process procedures describing user-level standard operations
	%new_hd
	The documentation for the Observing specialist as the main users of the observatory is under development in Confluence and can be found \href{https://rubinobs.atlassian.net/wiki/spaces/OST/overview | here] under \it{Training and Skills}
	
	\item Maintenance needs and procedures for all systems in use
	%new_hd
	The observatory has implemented a Computerized Maintenance Management System (CMMS). 
	It holds a growing number of the latest versions of repeatedly used maintenance procedures.
	
	\item A history of maintenance carried out during construction and commissioning
	%new_hd
	The CMMS allows for documenting the execution of maintenance activities and provides the history of all maintenance executions.

	\item System software documentation - including their operating versions, functionality, and interactions with other systems
	%new_hd
	
	\item The observatory feature-based scheduler algorithms and documentation for modification and refinement
	%new_hd
	The feature-based scheduler is realized as a Comandable SAL Component. Its code and documentation are stored in GitHub.
	
	\item A definition of initial delivered science data products (see previous sections)	
\end{itemize}

{\bf Note:} At the time of this update, the Project has recently set up a "Documentation Working Group".  This working group is responsible for defining the architecture of the delivered documentation repositories.

\subsection{Pre--Operations Interactions}

The final delivered documentation will be negotiated between the Rubin Construction Project and Rubin Operations.

\subsection{Artifacts for CCR}

See Criteria above.

\section{As-Built Record, Modifications, non-Compliance and Recommendations} \label{sec:recs}

\subsection{Operations Readiness Requirement}
The project team shall deliver all reports documenting the as-built hardware and software including: drawings, source code, modifications, compliance exceptions, and recommendations for improvement.

\subsection{Objectives:}

The objective of this readiness requirement is to ensure that the Construction Provide a record of the current state of the Rubin Observatory system at the time of its handover to the operations program.

A point of clarification: The Data Management science pipelines will be undergoing continuous development.  Commissioning will work with a specific release of the Rubin software stack.  The timing of which release will be used in commissioning will coincide with the readiness of the science camera -- LSSTCam.  Reporting of non-compliance of science pipeline functionality will be measured against this static release of the Rubin software stack.

\subsection{Criteria for Completeness}

The criteria for completeness of this requirement will be the production and delivery of the reports list in the artifacts below.  These reports shall document the final state of the observatory and non-compliance as known at the time of the conclusion of the commissioning phase of the project.  The reporting shall include recommendations for corrective measures for requirement found to be non-compliant and any recommendations for operational improvements based on the knowledge learned from the commissioning program.

\subsection{Pre--Operations Interactions}

The documentation provided by the Rubin Construction Project will conform to the document archiving architecture developed by the Rubm Operations team.

\subsection{Artifacts for ORR}

\begin{itemize}

	\item Report(s) documenting final as--built configuration of the hardware and software (see previous section)
	\item Report(s) documenting any modifications to the observatory that deviates for planned implements - {\it e.g.} field modifications made during the course of final commissioning activities;
	\item Report(s) of any non-compliance with system requirements and specifications;
	\item A report on the unresolved "punch list" items -- these are technical items that will need attention post construction completeness to improve operational performance, but extend beyond verification of system requirements; and
	\item A report from the Construction of recommendation for improvements based on results from commissioning.
	
\end{itemize}

\section{Rubin Operations Team Readiness} \label{sec:ops}

\subsection{Operations Readiness Requirement}

\begin{itemize}

	\item The Operations Team shall have a detailed operations plan approved by NSF and DOE.
	\item The Operations Team shall have a staffing plan with all roles in the operations plan filled with identified personnel. 
	\item The Operations Team shall demonstrate they can operate the delivered Rubin System
to efficiently capture, store, and process science quality images.
	
\end{itemize}

\subsection{Objectives}

The primary objective of this element of the ORR is that the Operations Team demonstrates that it 
is ready to smoothly continue running the full Rubin System as it exists at the end of the 
commissioning period. A successful initial phase of operations may include beginning the full 
Legacy Survey of Space and Time at the approved nightly schedule and cadence. 
It may also include other activities as necessary depending on the final outcome of commissioning. 
These could include special observing modes to enable Early Science and further 
development of detailed procedures for operations not done in commissioning but which 
do not prevent completion criteria from being satisfied.  

\subsection{Criteria for Readiness}

\begin{itemize}

	\item Demonstrate planning and staff for safe operations are in place. 
	\item The team should demonstrate that all needed roles are filled, or will be, with 
trained staff at the time of hand over to full operations.
	\item All Human Resources processes for on-boarding operations staff should be complete 
or ready by the date of handover as appropriate. Expatriate staff for Chile based deployments 
should have all necessary documents and requirements for work in Chile in place. Chilean staff 
should have any needed changes to their contracts made before operations begin. 
	\item An operations budget profile fully covering the needs of the observatory should be 
agreed to with the agencies in advance of full operations beginning.
	\item All supplies and non-labor capital items should be in place. 
	\item Contracts needed in year 1 for operations services or supplies should be in place. 
	\item Any in-kind contributions necessary for operations should be demonstrated to be in 
place and functioning at the level needed for year 1. Any systems handed over to operations 
from construction in advance of this review should be demonstrated to be functioning at the 
required level of performance.
	\item Demonstrate all needed advisory committees/structures are ready and in place.
	\item Demonstrate that all construction related documentation is captured 
in an operations documentation management system.
	\item Demonstrate ability to execute Alert Processing in the US DF including connectiing to community brokers.
	\item Demonstrate ability to execute Data Release Processing including delivery of data to
non US DF Data Facilities and ingest of data products from same for Data Access at USDF and Chile
DAC.  
	\item Demonstrate that a significant fraction of the community has been granted 
user accounts in the US DF, that the Rubin Science Platform supports their access and 
authorization and that they have been given suitable training or information to do science 
with the Rubin data products as they are delivered.

\end{itemize}

\subsection{Artifacts for ORR}

 As prelude: the Construction team will be responsible for creating sets/lists of topics/documents that fully describe the characteristics and performance of the Rubin systems, how to maintain them, how to operate them, and anything else critical for the Operations Team (initial survey of documents suggested date November 2020. The Operations Team will review these lists and identify anything that needs to be added (or removed) from those lists. A collaborative negotiation will be carried out with the Construction Team.

Final managing organization and agency approved Detailed Observatory Operations Plan, including:

\begin{itemize}
	\item Work Breakdown Structure;
	\item Activity based plans for each department;
	\item Milestones for each department though several years of operations;
	\item Performance metrics;
	\item Performance requirements;
	\item Maintenance Management plans;
	\item Fully populated staffing plan;
	\item Budget profile; and
	
\end{itemize}


\pagebreak

\appendix
% Include all the relevant bib files.
% https://lsst-texmf.lsst.io/lsstdoc.html#bibliographies
\section{References} \label{sec:bib}
\renewcommand{\refname}{} % Suppress default Bibliography section
\bibliography{local,lsst,lsst-dm,refs_ads,refs,books}

% Make sure lsst-texmf/bin/generateAcronyms.py is in your path
\pagebreak

\section{Acronyms} \label{sec:acronyms}
\addtocounter{table}{-1}
\begin{longtable}{p{0.145\textwidth}p{0.8\textwidth}}\hline
\textbf{Acronym} & \textbf{Description}  \\\hline

DOE & Department of Energy \\\hline
EPO & Education and Public Outreach \\\hline
LSE & LSST Systems Engineering (Document Handle) \\\hline
LSST & Legacy Survey of Space and Time (formerly Large Synoptic Survey Telescope) \\\hline
MREFC & Major Research Equipment and Facility Construction \\\hline
NSF & National Science Foundation \\\hline
ORR & Operations Readiness Review \\\hline
SE & System Engineering \\\hline
SRD & LSST Science Requirements; LPM-17 \\\hline
\end{longtable}

% If you want glossary uncomment below -- comment out the two lines above
%\printglossaries

\end{document}
