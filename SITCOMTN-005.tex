\documentclass[SE,authoryear,toc]{lsstdoc}
% lsstdoc documentation: https://lsst-texmf.lsst.io/lsstdoc.html
\input{meta}

% Package imports go here.

% Local commands go here.

%If you want glossaries
%\input{aglossary.tex}
%\makeglossaries

\title{Operations Readiness Criteria}

% Optional subtitle
% \setDocSubtitle{A subtitle}

\author{%
William O'Mullane
}

\setDocRef{SITCOMTN-005}
\setDocUpstreamLocation{\url{https://github.com/lsst-sitcom/sitcomtn-005}}

\date{\vcsDate}

% Optional: name of the document's curator
% \setDocCurator{The Curator of this Document}

\setDocAbstract{%
This technote collects together the elements that constitute criteria for Operations Readiness of the Rubin Observatory
}

% Change history defined here.
% Order: oldest first.
% Fields: VERSION, DATE, DESCRIPTION, OWNER NAME.
% See LPM-51 for version number policy.
\setDocChangeRecord{%
  \addtohist{1}{YYYY-MM-DD}{Unreleased.}{William O'Mullane}
}


\begin{document}

% Create the title page.
\maketitle
% Frequently for a technote we do not want a title page  uncomment this to remove the title page and changelog.
% use \mkshorttitle to remove the extra pages

% ADD CONTENT HERE
% You can also use the \input command to include several content files.

\appendix
% Include all the relevant bib files.
% https://lsst-texmf.lsst.io/lsstdoc.html#bibliographies
\section{References} \label{sec:bib}
\renewcommand{\refname}{} % Suppress default Bibliography section
\bibliography{local,lsst,lsst-dm,refs_ads,refs,books}

% Make sure lsst-texmf/bin/generateAcronyms.py is in your path
\section{Acronyms} \label{sec:acronyms}
\addtocounter{table}{-1}
\begin{longtable}{p{0.145\textwidth}p{0.8\textwidth}}\hline
\textbf{Acronym} & \textbf{Description}  \\\hline

3D & Three-dimensional \\\hline
AI & Artificial Intelligence \\\hline
AP & Alert Production \\\hline
AURA & Association of Universities for Research in Astronomy \\\hline
AVM & Audio--Visual Management \\\hline
B & Byte (8 bit) \\\hline
CAD & Computer Aided Design \\\hline
CCD & Charge-Coupled Device \\\hline
CMMS & Computerized Maintenance Management System \\\hline
CSA & Cooperative Support Agreement \\\hline
CSC & Commandable SAL Component \\\hline
DAC & Data Access Center \\\hline
DDF & Deep Drilling Fields \\\hline
DF & Data Facility \\\hline
DIA & Difference Image Analysis \\\hline
DIMM & Differential Image Motion Monitor \\\hline
DM & Data Management \\\hline
DMS & Data Management Subsystem \\\hline
DMSR & DM System Requirements; LSE-61 \\\hline
DOE & Department of Energy \\\hline
DPDD & Data Product Definition Document \\\hline
DR1 & Data Release 1 \\\hline
DRP & Data Release Production \\\hline
EFD & Engineering and Facility Database \\\hline
EPO & Education and Public Outreach \\\hline
FMEA & failure modes and effect analysis \\\hline
FPA & Focal Plane Array \\\hline
FRACAS & Failure Reporting, Analysis and Corrective Action System \\\hline
FWHM & Full Width at Half-Maximum \\\hline
GAIA & Global Astrometric Interferometry for Astrophysics \\\hline
GUI & Graphical User Interface \\\hline
LDM & LSST Data Management (Document Handle) \\\hline
LOVE & LSST Operations Visualization Environment \\\hline
LPM & LSST Project Management (Document Handle) \\\hline
LSE & LSST Systems Engineering (Document Handle) \\\hline
LSR & LSST System Requirements; LSE-29 \\\hline
LSST & Legacy Survey of Space and Time (formerly Large Synoptic Survey Telescope) \\\hline
LVV & LSST Verification and Validation \\\hline
MPC & Minor Planet Center \\\hline
MREFC & Major Research Equipment and Facility Construction \\\hline
NSF & National Science Foundation \\\hline
ORR & Operations Readiness Review \\\hline
OSS & Observatory System Specifications; LSE-30 \\\hline
PSF & Point Spread Function \\\hline
QA & Quality Assurance \\\hline
QC & Quality Control \\\hline
RAT & Rubin Auxiliary Telescope \\\hline
RSP & Rubin Science Platform \\\hline
RTN & Rubin Technical Note \\\hline
SAL & Service Abstraction Layer \\\hline
SDK & Software Development Kit \\\hline
SDQA & Science Data Quality Assessment \\\hline
SE & System Engineering \\\hline
SITCOMTN & System Integration, Test and Commissioning Technical Note \\\hline
SLAC & SLAC National Accelerator Laboratory \\\hline
SQuaSH & Science Quality Analysis Harness \\\hline
SRD & LSST Science Requirements; LPM-17 \\\hline
SSP & Solar System Processing \\\hline
SV & Science Validation \\\hline
T\&S & Telescope and Site \\\hline
TBC & To Be Confirmed \\\hline
TBD & To Be Defined (Determined) \\\hline
TMA & Telescope Mount Assembly \\\hline
UPS & uninterruptible power supply \\\hline
US & United States \\\hline
USDF & United States Data Facility \\\hline
WEB & World Wide Web \\\hline
WFD & Wide Fast Deep \\\hline
deg & degree; unit of angle \\\hline
\end{longtable}

% If you want glossary uncomment below -- comment out the two lines above
%\printglossaries





\end{document}
