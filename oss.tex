\section{Observatory System Specifications (LSE-30) Verifcation}  \label{sec:oss}

\subsection{Operations Readiness Requirement}
The project team shall demonstrate that the integrated LSST systems (Camera, Telescope \& Site and Data Management subsystems) as well as the Education and Public Outreach (EPO) system have met the technical specifications enumerated in the LSST Observatory System Specifications (\citeds{LSE-30}).

%new_hd
The requirements in \citeds{LSE-30} have been marked according to the CCR where they can be earliest verified. 
The distribution between the CCRs is shown in Figure~\ref{CCRs_overview}.

\begin{figure}[htbp]
\begin{center}
%\includegraphics[width=1\textwidth]{./XXX.png}
\caption{Distribution of the the LSE30 requirement verification over the course of the CCRs}
\label{LSE30_CCRdistribution}
\end{center}
\end{figure}


\subsection{Objectives}
The main objective of this Operations Readiness Requirement is to verify the system specifications in the OSS (\citeds{LSE-30}) are proven and well documented. The OSS is essentially the highest-level document describing the basic LSST system technical architecture. It contains sections derived from the OSS on the following broad topics:

\begin{itemize}
\item System Composition and Constraints

\item Common System Functions and Performance, including:

	\begin{itemize}
		\item System Control
		%new_hd
		The System Control is implemented by combining a Service Abstraction Layer (SAL) and a number of Commandable SAL Components (CSC).
		A CSC represents each System and Subsystem in the observatory.
		Each CSC has a well-defined interface with the SAL. All other CSCs are required to comply with the definition of the interface.
		Therefore, the interface definitions are handled as requirements and verified as such. 
		Each interface requirement is verified through unit testing on the teststands at each new release and with the hardware during system usage.  
		Artifact?
		
		\item System Monitoring and Diagnostics
		As part of the communication between the CSCs, messages with Commands, Events, and Telemetry are exchanged. 
		These are stored in real-time in the Engineering database and can be displayed through Chronograph, Rubin TV, and others.
		To verify these efforts, we demonstrate the capabilities during the observatory visit. 
				
		\item System Maintenance
		Maintenance started as soon as the Observatory started to use components that needed maintenance, such as generators.
		We have implemented a Computerized Maintenance Management System (CMMS) and connected it to our work management system (Jira) 
		
		\item System Availability
		
		\item System Time References
		
	\end{itemize}

\item Detailed Specifications:

	\begin{itemize}
		\item Science and Bulk Data
		\item Optical System
		\item System Throughput
		\item Camera System
		\item Photometric Calibration
		\item System Timing and Dynamics
	\end{itemize}
	
\item Education and Public Outreach

\end{itemize}

\subsection{Criteria for Completeness}
Compliance with this objective will follow the process defined in the Verification and Validation Process document (\citeds{LSE-160}) and associated documentation.  
All technical specifications in the OSS (\citeds{LSE-30}) and LSR (\citeds{LSE-29}) are expected to be met at the end of construction.

\subsection{Pre--Operations Interaction}
None. Unless there are non-compliance issues with the ORR requirements and specifications.

\subsection{Artifacts for Completion}

\begin{itemize}
 
	\item Verification matrix containing entries for all OSS requirements and specifications.  The verification method: inspection, demonstration, analysis or test shall be identified for every OSS requirement.  Final compliance status will be included.
	\item Analysis reports where the verification method has been identified as "test" or "analysis".
	\item Non-compliance reports.

\end{itemize}