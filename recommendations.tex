\section{As-Built Record, Modifications, non-Compliance and Recommendations} \label{sec:recs}

\subsection{Operations Readiness Requirement}

The project team shall deliver all reports documenting the as-built hardware and software, including drawings, source code, modifications, compliance exceptions, and recommendations for improvement.

\subsection{Objectives:}

The objective of this readiness requirement is to ensure that the Construction Project provides a record of the current technical state of the Rubin Observatory system and that the knowledge transfer necessary for operations and further development of the Rubin Observatory is provided in a form that allows the operations team to conduct the 10-year planned survey.

A point of clarification: The Data Management science pipelines will be undergoing continuous development.  Commissioning will work with a specific release of the Rubin software stack.  The timing of which release will be used in commissioning will coincide with the readiness of the science camera -- LSSTCam.  Reporting of science pipeline functionality non-compliance will be measured against this static release of the Rubin software stack.

\subsection{Criteria for Completeness}

The criteria for completeness of this requirement will be the production and delivery of the reports listed in the artifacts below.  These reports shall document the final state of the observatory and non-compliance as known at the time of the conclusion of the commissioning phase of the project.  The reporting shall include recommendations for corrective measures for requirements found to be non-compliant and any recommendations for operational improvements based on the knowledge learned from the commissioning program.

Specific items include:

\begin{itemize}
	\item A configuration management plan for observatory-wide software systems

	\item A clearly defined and documented architecture and implementation for the Project's varied documentation.  This includes:
	\begin{itemize}
		\item Design documents describing the technical implementation for all major subsystems
		%new_hd
		\textrightarrow This can already be found in DocuShare.
		A new DocuShare structure for Operations is under development \href{https://docushare.lsst.org/docushare/dsweb/View/Collection-15974}{here}

		\item 3D CAD models and fabrication drawings
		%new_hd
		These models are stored in a dedicated SolidWorks server. Solidworks is the program the project chose to develop mechanical designs.
		Vendors could choose their preferred drawing program. Drawings made with other programs were converted, and the original version is archived in Docushare.

		\item Operating software versions and their documentation
		%new_hd
		Software written at or modified at the Observatory is documented, reviewed and version-controlled on GitHub.

		\item Definition of delivered data properties
		%new_hd
		Here is the can find the \href{ls.st/dpdd}{“Data Products Definition Document”}.
		\item Software source codes and their documentation
		%new_hd
		Software and its documentation, written by or modified at the Observatory, are documented, reviewed, and version-controlled on GitHub.

		\item As-built drawings, diagrams and metrology
		%new_hd
		This is stored in DocuShare.
		The test results of the metrology for verification purposes are added to the execution of test cases in our test manager (Zephyr Scale) connected to Jira.

		\item Clear traceability between the systems requirements and how they were verified
		%new_hd
		Requirements are either verified by results captured in test cases or lower-level requirements.
		The test cases are grouped by test cycles, and test cycles are grouped by test plans.
		There is full traceability between the requirement and its verification.

		\item Clear traceability and documentation for deviations/waivers to the systems requirements
		%new_hd
		 Deviations/waivers are traced to the impacted requirements
		 On the other hand, deviations/waivers are traced to the corresponding change request and the related processes in the Change Control Board.

		\item Verification artifacts, including test results, analyses, and inspection reports
		%new_hd
		Verification artifacts are connected to test cases.
		Either the test case execution includes the information directly in the test steps or is attached as a file to the test case.
		The code needed to reproduce the results is stored in GitHub when available.

		\item FRACAS reportable failures during integration, verification, and commissioning
		%new_hd
		The FRACAS system is implemented as a Jira project and has been actively updated since the early integration phase.

		\item Hazard Analysis including hazard mitigation verifications
		%new_hd
		Hazards have been imported into the Jira system as part of the LVV project.  The Hazards are analyzed during a weekly meeting.
		Hazard mitigations are suggested by the meeting members, implemented by the summit technical team, and documented by the systems engineering team.
		The hazard mitigation artifact is added to the ticket, and a safety specialist reviews the measures and evaluates the residual risk.

		\item FMEA for all major subsystems
		%new_hd
		Failure modes are registered as they are experienced in the FRACAS.
		Critical lifts have an FMEA attached, and failure modes are mitigated as much as possible before.

	\item A WEB-based (and associated document) roadmap/directory for the Project's document repositories (see above).
	%new_hd

\end{itemize}

{\bf Note:} At the time of this update, the Project has recently set up a "Documentation Working Group".  This working group is responsible for defining the architecture of the delivered documentation repositories.

\subsection{Pre--Operations Interactions}

The documentation provided by the Rubin Construction Project will conform to the document archiving architecture developed by the Rubin Operations team.  The final delivered documentation will be negotiated between the Rubin Construction Project and Rubin Operations.

\subsection{Artifacts for Completion}

\begin{itemize}

	\item Report(s) documenting final as--built configuration of the hardware and software (see previous section)
	\item Report(s) documenting any modifications to the observatory that deviate from planned implements - {\it e.g.} field modifications made during the course of final commissioning activities;
	\item Report(s) of any non-compliance with system requirements and specifications;
	\item A report on the unresolved "punch list" items -- these are technical items that will need attention post construction completeness to improve operational performance but extend beyond verification of system requirements; and
	\item A report from the Construction of recommendations for improvements based on results from commissioning.
\end{itemize}

\end{itemize}
