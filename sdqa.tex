\section{Science Data Quality Assessment}  \label{sec:sdqa}


\subsection{Operations Readiness Requirement}

The project team shall demonstrate that the integrated LSST system can monitor and assess the quality of all data as it is being collected.

\subsection{Objectives} 

Science Data Quality Assessment is made up of a comprehensive system of tools to monitor and assess quality of all data as it is being collected including raw and processed data. The suite of tools have been designed to collect, analyze and record required information to assess the data quality and make that information available to a variety of end users; observatory specialist, observatory scientists, downstream processing, the science planning/scheduling process and science users of the data. 

The fast cadence of data collection requires highly automated data diagnostic and analysis methods (such as data mining techniques for finding patterns in large datasets, and various machine learning regression techniques). he Science Data Quality Assessment is mostly be automated, however it includes human-intensive components allowing further investigation and visualization of SDQA status.

Data quality assessment for Rubin must be carried out at a variety of cadences, which have different goals:

\begin{itemize}

	\item Near real-time assessment of whether the data is scientifically useful;
	\item Monitoring telemetry and imaging data to track the state of the integrated observatory, including the telescope, camera, networks and other supporting systems;
	\item Analysis of the prompt processing properties and performance to determine if the alerts stream meets its requirements; and
	\item Analysis of the data release processing properties and performance to determine if the static sky processing meets its requirements.
	
\end{itemize}

By the time we make a data release the accumulated data quality analysis must be made available as part of the release artifacts.

\subsubsection{Near Real--time Monitoring \& Assessment of the raw data quality} 

The quality assessment of the raw image data combines the results from the state of the telescope, the camera (see below) and technical properties of the images.  Each will be analyzed as it is taken to a measure its technical properties both on the at the Summit Facility using the LSSTCam Diagnostic cluster and from properties determined during the prompt processing for alert production.  Performance properties will be based on measurements and characteristics derived from the images themselves and from daily calibration data, these include:

\begin{itemize}

	\item Sensor readnoise, bias and gain variations, bitwise integrity etc...  from the CCD data;
	\item Properties of the measured PSF, based on the three second moments, or equivalently effective FWHM, e1, e2;
	\item Measured sky background level over the full FPA at amplifier level resoution;
	\item Measured source positions and errors relative to a reference catalog ({\it e.g. GAIA}) to monitor FPA stability and pointing accuracy; and
	\item Measured source fluxes and errors relative to a reference catalogue ({\it e.g. GAIA}) to monitor system throughput, sensitivity and algorithm processing.
	
\end{itemize}

At a minimum, these metrics enable the Project to determine if the data are within performance parameters to label the visit as "good".   Tooling will be provided by the Construction Project that enable users to monitor trends in these quantities ({it. e.g.} as a function of time and where the telescope is pointing and as a function of position in the focal plane.  A reference set of tools will initially be provided by the LOVE interface along with more detailed analysis tooling (as described below).  In some cases, data from the Rubin Auxiliary Telescope (RAT) will be used to interpret trends the LSSTCam data.  The quality analysis needed to determine that the RAT is taking sufficiently good data will use the same tooling as provided for the main survey data.

\subsubsection{Longer Term Assessment}

TBD

\subsubsection{Assessing the quality of the processed data}

The information of the processed data relies on the calibration data products and the pipeline properties. In other words, the data assessment at this stage shall include the correction of the systematic errors. 

\subsection{SDQA Tools for analysis}

Science Data Quality Assessment will rely on a suite of tools including as the electronic logging, the engineering facility database (EFD), and the Rubin Science Platform (RSP).  There is also a complementary set data visualization tools to facilitate the understanding of the correlation between the data quality and the observatory state. 

These tools include:

\begin{itemize}

	\item Rubin Science Platform (RSP) -- used for investigative ad--hoc analysis (\citeds{LSE-319});  the RSP itself through it's web based porthole and Jupyter Lab interface provides significant visualization capabilities;
	\item Engineering Facility Database -- accessible through science platform and pre-defined dashboards;
	\item LOVE - LSST Observing Visualization Environment used to have standardized dashboards and visualization of the system state;
	\item SQuaSH - the Science Quality System Harness (\citeds{sqr-009})

\end{itemize}

\subsection{Criteria for Completeness Description}

The SDQA shall monitor and record the properties of the system error budget tree, including image quality and throughput, and define pass or fail status at each of the primary entries entries.   These include the following terms of the image quality: 

\begin{itemize}

	\item PSF FHWM;
	\item PSF shape ellipticity as described by second moments;
	\item System wavefront measurements for each visit; and
	\item Throughput measurements over the entire field of view.
		
\end{itemize}

Tooling for evaluating SDQA shall demonstrate the ability to display performance on a visit by visit basis as well as being able to show the history of performance metric over a user defined span of time.

\subsection{Pre-Operations Interactions}
The pre-operation interaction include training the observing specialists to understand errors 

\subsection{Artifacts for ORR}

\begin{itemize}

	\item Demonstrated functional tool kit as described above;
	\item Code validation tool kit to quantify software performance;
	\item Derived reporting from the Science Verification/Validation survey(s)
	
\end{itemize}





